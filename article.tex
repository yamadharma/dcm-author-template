{%
\selectlanguage{english}

%\graphicspath{{image/zaryadov/}}

\udc{519.872, 519.217}


\title{Towards the Analysis of the Queuing System Operating in~the~Random Environment with Resource Allocation}


\author[1,2]{Ivan S. Zaryadov}
\author[1]{Vladimir V. Tsurlukov}
\author[1]{H. Viana Carvalho Cravid}
\author[1]{Anna A. Zaytseva}
\author[1]{Tatiana A. Milovanova}

\address[1]{
  Department of Applied Probability and Informatics\\
  Peoples' Friendship University of Russia (RUDN University)\\
  6, Miklukho-Maklaya str., Moscow, 117198, Russian Federation}

\address[2]{Institute of Informatics Problems, FRC CSC RAS\\
  IPI FRC CSC RAS, 44-2 Vavilova Str., Moscow 119333, Russian
  Federation}


% \received{\formatdate{09}{10}{2018}}


\begin{authordescription}
% \authorfull[russian]{Зарядов Иван Сергеевич}
\authorfull[english]{Zaryadov, Ivan S.}
% \authordegree[russian]{кандидат физико\-/математических наук}
\authordegree[english]{Candidate of Physical and Mathematical Sciences}
% \authorpost[russian]{доцент кафедры прикладной информатики и теории вероятностей РУДН, старший научный сотрудник ИПИ ФИЦ ИУ РАН}
\authorpost[english]{assistant professor of   Department of Applied
   Probability  and Informatics of  Peoples' Friendship
    University of Russia (RUDN University); Senior Researcher of
    Institute of Informatics Problems of 
  Federal Research Center ``Computer Science and Control'' 
  Russian Academy of Sciences}
\email{zaryadov-is@rudn.ru}
\orcid{0000-0002-7909-6396}
\phone{+7(495)9550927}
\end{authordescription}


% \begin{authordescription}
% \authorfull{Зарядов Иван Сергеевич}
% \authordegree{кандидат физико\-/математических наук}
% \authorpost{доцент кафедры прикладной информатики и теории вероятностей  РУДН, старший научный сотрудник ИПИ РАН}
% \email{zaryadov-is@rudn.ru}
% \phone{+7(495)9550927}
% \end{authordescription}

\begin{authordescription}
% \authorfull[russian]{Цурлуков Владимир Владимирович}
\authorfull[english]{Tsurlukov, Vladimir V.}
% \authordegree{}
% \authorpost[russian]{магистр кафедры прикладной информатики и теории вероятностей РУДН}
\authorpost[english]{master's degree student of   Department of Applied
  Probability  and Informatics of  Peoples' Friendship
  University of Russia (RUDN University)}
\email{dober.vvt@gmail.com}
\phone{+7(495)9550927}
\end{authordescription}

\begin{authordescription}
% \authorfull[russian]{Виана Карвалью Кравид Илкиаш}
\authorfull[english]{Viana, Carvalho Cravid H.}
% \authordegree{}
% \authorpost[russian]{магистр кафедры прикладной информатики и теории вероятностей РУДН}
\authorpost[english]{master's degree student   of   Department of Applied
   Probability  and Informatics of  Peoples' Friendship
    University of Russia (RUDN University)}
\email{hilvianamat1@gmail.com }
\phone{+7(495)9550927}
\end{authordescription}

\begin{authordescription}
% \authorfull[russian]{Зайцева Анна Андреевна}
\authorfull[english]{Zaytseva, Anna A.}
%\authordegree{}
% \authorpost[russian]{магистр кафедры прикладной информатики и теории вероятностей РУДН}
\authorpost[english]{master's degree student of     Department of Applied
   Probability  and Informatics of  Peoples' Friendship
    University of Russia (RUDN University)}
\email{anna-z96@mail.ru }
\phone{+7(495)9550927}
\end{authordescription}

\begin{authordescription}
% \authorfull[russian]{Милованова Татьяна Александровна}
\authorfull[english]{Milovanova, Tatiana A.}
% \authordegree[russian]{кандидат физико\-/математических наук}
\authordegree[english]{Candidate of Physical and Mathematical Sciences}
% \authorpost{старший преподаватель кафедры прикладной информатики и теории вероятностей  РУДН}
\authorpost[english]{lecturer of Department of Applied
  Probability  and Informatics of  Peoples' Friendship
  University of Russia (RUDN University)}
\email{milovanova-ta@rudn.ru}
\orcid{0000-0002-9388-9499}
\phone{+7(495)9550927}
\end{authordescription}

\abstracts{ The mathematical model of the system, that consists of a
  storage device and several homogeneous servers and operates in a
  random environment, and provides incoming applications not only
  services, but also access to resources of the system, is being
  constructed.  The random environment is represented by two
  independent Markov processes.  The first of Markov processes
  controls the incoming flow of applications to the system and the
  size of resources required by each application. The incoming flow is
  a Poisson one, the rate of the flow and the amount of resources
  required for the application are determined by the state of the
  external Markov process. The service time for applications on
  servers is exponential distributed. The service rate and the maximum
  amount of system resources are determined by the state of the second
  external Markov process. When the application leaves the system, its
  resources are returned to the system. In the system under
  consideration, there may be failures in accepting incoming
  applications due to a lack of resources, as well as loss of the
  applications already accepted in the system, when the state of the
  external Markov process controlling the service and provision of
  resources changes.  A random process describing the functioning of
  this system is constructed. The system of equations for the
  stationary probability distribution of the constructed random
  process is presented in scalar form. The main tasks for further
  research are formulated.  }

\keywords{queuing system, random environment, Markov modulated Poisson
  process, Markov modulated service process, resource allocation}

%\thanks{The publication has been prepared with the support of the
%  ``RUDN University Program 5-100'' and funded by RFBR according to
%  the research projects No.~18-07-00692 and No.~16-07-00766.}



% \alttitle{К анализу системы массового обслуживания с ресурсами, функционирующей в случайном окружении}


% \altauthor[1,2]{И. С. Зарядов}
% \altauthor[1]{В. В. Цурлуков}
% \altauthor[1]{И. Виана Карвалью Кравид}
% \altauthor[1]{А. А. Зайцева}
% \altauthor[1]{Т. А. Милованова}

% \altaddress[1]{Кафедра прикладной информатики и теории вероятностей\\
%   Российский университет дружбы народов\\
%   ул. Миклухо-Маклая, д.6, Москва, Россия, 117198}

% \altaddress[2]{Институт проблем информатики\\
%   Федеральный исследовательский центр <<Информатика и управление>> РАН \\
% %  Российской академии наук\\
%   ул. Вавилова, д.~44, кор.~2, Москва, Россия, 119333}





% \altabstracts{ Строится математическая модель системы, состоящей из
%   накопителя и нескольких однородных приборов, функционирующей в
%   случайном окружении и предоставляющей поступающим заявкам помимо
%   обслуживания еще и доступ к ресурсам. Случайное окружение
%   представлено двумя независимыми марковскими процессами, управляющими
%   поступлением заявок в систему и обслуживанием заявок. В систему
%   поступает пуассоновский поток заявок, интенсивность поступления и
%   объем ресурсов, необходимый заявке при обслуживании, определяются
%   состоянием внешнего марковского процесса. Время обслуживания заявок
%   на приборах подчинено экспоненциальному распределению. Интенсивность
%   обслуживания и максимальный объем ресурсов системы определяются
%   состоянием второго внешнего марковского процесса. При окончании
%   обслуживания заявки, занятые ею ресурсы возвращаются в систему.  В
%   рассматриваемой системе возможны отказы в приеме поступающих заявок
%   из-за нехватки ресурсов, а также возможны потери уже принятых в
%   систему заявок при изменении состояния внешнего марковского
%   процесса, управляющего обслуживанием и предоставлением
%   ресурсов. Построен случайные процесс, описывающий функционирование
%   данной системы. Представлена в скалярной форме система уравнений для
%   стационарного распределения вероятностей построенного случайного
%   процесса. Сформулированы основные задачи для дальнейшего
%   исследования.  }

% \altkeywords{система массового обслуживания, случайное окружение,
%   MMPP, предоставление ресурсов}


\maketitle

\section{Introduction}
\label{sec:intro}

The mathematical model of the analysis of the functioning of modern
telecommunication systems must take into account the influence of
external factors, which may be realized within the framework of the
queuing theory (the theory of
teletraffic)\cite{Boch,bash1,MMAP-book,5} with the help of arrival
and/or service processes controlled by some external random
process. The application of the Markov modulated arrival process
(MMAP), Markov modulated service process
(MMSP))~\cite{Neuts3,Neuts4,Neuts5,Neuts6,4,MMAP-book} allows us to
construct not only the adequate mathematical model, but also to obtain
good analytical results for different
tasks~\cite{MMPP-1998,MMPP-2005,vish,koz1,koz2,koz3,raz0,raz1,raz2,raz3,sam1,sam2,6b}

The mathematical modeling of modern telecommunication systems when
incoming applications in addition to services also require some fixed
or variable volume of
resources~\cite{sop1,sop3,sop4,sop5,sop6,sop7,sop8} is the actual
problem.

We will try to apply Markov modulated Poisson process (MMPP)
theory~\cite{Neuts3,Neuts4,Neuts5,Neuts6,4} to construct the
mathematical model of the system, that consists of a storage device
and several homogeneous servers and operates in a random environment,
provides incoming applications not only services, but also access to
resources of the system, is being constructed. The random environment
is represented by two independent Markov processes. The first of
Markov processes controls the incoming flow of applications to the
system and the size of resources required by each application. The
service rate and the maximum amount of system resources are determined
by the states of the second external Markov process.  The initial
stages of this study were presented in~\cite{13}. The system of
equations for the stationary probability distribution of the random
process, describing the behavior of the system, is the main goal of
this part of the research.



\section{System description} 
\label{sec:base-section}

We will consider the queueing system $MMPP_2 |MMSP_2 |n|r|R_1,R_2$
(according to Kendall--Basharin notation\cite{Boch}), functioining in
the random environment (Markov modulated Poisson arrival process and
Markov modulated service process), with $1\leqslant  n<\infty$ homogeneous
servers and the buffer of $r\leqslant  \infty$ capacity.


The random environment is present by two-state Markov process (MP)
$\eta_1 (t)$, which control the incoming Poisson process.  If the
external Markov process $\eta_1 (t)$ is in state $1$ then the rate of
incoming Poisson process is $\lambda_1$ and each arriving application
requires the fixed $k_1$ amount of system resources. If the MP
$\eta_1 (t)$ is in state $2$ the each application arrives according
the Poisson law with the rate $\lambda_2$ and requires the fixed
amount of system resources of size $k_2$.

The second external two-state Markov process $\eta_2 (t)$ controls the
service process on system servers and the maximum amount of system
resources. If MP $\eta_2 (t)$ is in the state $1$, then the maximum
value of system resources is $R_1<\infty$, the service time of an
application (on each of $n$ homogeneous servers) is subject to the
exponential distribution with the rate $\mu_1$. If MP $\eta_2 (t)$ is
in the state $2$, then the amount of system resources $R_2$ is
unlimited, the service time of an application (on each of $n$
homogeneous servers) is subject to the exponential distribution but
with the rate $mu_2$.

The transitions of Markov processes $\eta_1$ and $\eta_2$ from one
state to another are determined by the corresponding infinitesimal
matrices $\Lambda=\left(\lambda_{ij}\right)_{i,j=1,2}$ and
$M=\left(\mu_{ij}\right)_{i,j=1,2}$.

After the end of the service each application returns to the system
the resources, occupied by this application.

The functioning of the system may be defined by the multidimensional
random process
$\zeta(t)=\{\mathbf{\xi}_1 (t), \mathbf{\xi}_2 (t),R(t),\eta_1
(t),\eta_2 (t)\}$, where random process
$\mathbf{\xi}_1 (t)=\left(\xi_{1s}(t),\xi_{1q}(t)\right)$ describes
the number of applications with demand on $k_1$ amount of resources
(applications of the first type) on the servers ($\xi_{1s}(t)$) and in
the buffer ($\xi_{1q}(t)$) at the time moment $t$.  Respectively, the
random process
$\mathbf{\xi}_2 (t)=\left(\xi_{2s}(t),\xi_{2q}(t)\right)$ --- the
number of application with demand on $k_2$ amount of resources
(applications of the second type) on the servers ($\xi_{2s}(t)$) and
in the buffer ($\xi_{2q}(t)$) at the time moment $t$. $R(t)$ --- the
available at time $t$ amount of system resources. If the state of the
Markov process $\eta_2$ is $1$, then
$R(t)=\max(0, R_1 - k_1\mathbf{\xi}_1(t)\mathbf{1} - k_2
\mathbf{\xi}_2 (t)\mathbf{1})$, if the state of the Markov process
$\eta_2$ is $2$ then $R(t)=R_2=\infty$.

If the amount of the system resources $R(t)$ at the moment of the new
application arrival is less then $k_1$ (for the first type
application) or $k_2$ (for the second type application) amount of
resources needed in addition to service (i.e. $R(t)<k_1$ or
$R(t)<k_2$), then the incoming application is lost. Also the accepted
to the system applications may be dropped from the buffer due to the
transition Markov chain $\eta_2$ from state $2$ with unlimited amount
$R_2$ of system resources to the state $1$ with limited amount of
resources
$R(t)=R_1 - k_1\mathbf{\xi}_1(t)\mathbf{1} - k_2 \mathbf{\xi}_2
(t)\mathbf{1}$.

In order to avoid downtime of servers it is supposed that the maximum
value of system resources $R_1<\infty$ is sufficient for all sersers
to be occupied, that is $R_1\geqslant  n\cdot\max(k_1,k_2)$.


The goal of this paper is to derive the system of equations for random
process $\zeta(t)$ steady-state probability distribution. The main
goals of the study as a whole are to obtain main time-probability
characteristics of the system as for this general case (also for the
case when the maximum values of system resources are finite, but
different for all states of governing external Markov process), and
for special cases of only one external governing Markov process.


%%%%%%%%%%%%%%%%%%%%%%%%%%%%%%%%%%%%%%%%%%%%%%%%%%%%%

\section{The steady-state probability distribution. The system of
  equations (scalar form)}
\label{sec:base-steady}

The set $\mathcal{X}$ of states of the random process
$\zeta(t)=\{\mathbf{\xi}_1 (t), \mathbf{\xi}_2 (t),R(t),\eta_1
(t),\eta_2 (t)\}$ may be presented as
$\mathcal{X}=\{(i_s; i_q),(j_s; j_q),R_1(i_s+i_q;j_s+j_q)|R_2,l,m\}$.
Here, $i_s$ and $i_q$ ($0\leqslant i_s\leqslant n$, $i_q\geqslant 0$)
are numbers of the first type applications on servers ($i_s$) and in
the buffer ($i_q$); $j_s$ and $j_q$ ($0\leqslant j_s\leqslant n$,
$j_q\geqslant 0$) are numbers of the second type applications on
servers ($j_s$) and in the buffer ($j_q$).  It should be noted that
$0\leqslant i_s + j_s\leqslant n$. The argument $l=1,2$ describes the
state of the external Markov process $\eta_1$ as well as the $m=1,2$
--- the state of the Markov process
$\eta_2$. $R_1(i_s+i_q;j_s+j_q)=R_1-(i_s+i_q)k_1-(j_s+j_q)k_2$ --- the
current amount of the system resources in the state $1$ of Markov
process $\eta_2$.


In the case of the buffer of unlimited capacity, the entire set of
states can be divided into $10$ subsets corresponding to the following
states:
\begin{enumerate}
\item the system is empty --- the
  states $\left\{(0; 0),(0; 0), R_1(0;0),1,1\right\}$,
  $\{(0; 0),(0; 0), R_1(0;0),\\* 2,1\}$,
  $\left\{(0; 0),(0; 0), R_2,1,2\right\}$,
  $\left\{(0; 0),(0; 0), R_2,2,2\right\}$;
\item there are only applications of the first type in the system, not
  all servers are occupied, the buffer is empty ---
  $\left\{(i_s; 0),(0; 0), R_1(i_s;0),1,1\right\}$,
  $\{(i_s; 0),(0; 0), R_1(i_s;0),\\* 2,1\}$,
  $\{(i_s; 0),(0; 0), R_2,1,2\}$, $\{(i_s; 0),(0; 0), R_2,2,2\}$,
  $1\leqslant  i_s < n$;
\item there are only applications of the first type in the system, all
  servers are occupied, the buffer is empty ---
  $\left\{(n; 0),(0; 0), R_1(n;0),1,1\right\}$,
  $\left\{(n; 0),(0; 0), R_1(n;0),2,1\right\}$,
  $\left\{(n; 0),(0; 0), R_2,1,2\right\}$,
  $\left\{(n; 0),(0; 0), R_2,2,2\right\}$;
\item there are only applications of the first type in the system, all
  servers are occupied, the buffer is not empty ---
  $\left\{(n; i_q),(0; 0), R_1(n+i_q;0),1,1\right\}$,
  $\{(n; i_q),(0; 0), \\*R_1(n+i_q;0),2,1\}$,
  $\left\{(n; i_q),(0; 0), R_2,1,2\right\}$,
  $\left\{(n; i_q),(0; 0), R_2,2,2\right\}$, $i_q\geqslant  1$;
\item there are only applications of the second type in the system,
  not all servers are occupied, the buffer is empty ---
  $\left\{(0; 0),(j_s; 0), R_1(0;j_s),1,1\right\}$,
  $\{(0; 0),(j_s; 0), R_1(0;j_s),\\* 2,1\}$,
  $\{(0; 0),(j_s; 0), R_2,1,2\}$, $\{(0; 0),(j_s; 0), R_2,2,2\}$,
  $1\leqslant  j_s <n$;
\item there are only applications of the second type in the system,
  all servers are occupied, the buffer is empty ---
  $\left\{(0; 0),(n; 0), R_1(0;n),1,1\right\}$,
  $\left\{(0; 0),(n; 0), R_1(0;n),2,1\right\}$,
  $\left\{(0; 0),(n; 0), R_2,1,2\right\}$,
  $\left\{(0; 0),(n; 0), R_2,2,2\right\}$;
\item there are only applications of the second type in the system,
  all servers are occupied, the buffer is not empty ---
  $\left\{(0; 0),(n; j_q), R_1(0;n+j_q),1,1\right\}$,
  $\{(0; 0),(n; j_q), \\*R_1(;n+j_q),2,1\}$,
  $\left\{(0; 0),(n; j_q), R_2,1,2\right\}$,
  $\left\{(0; 0),(n; j_n), R_2,2,2\right\}$, $j_q\geqslant  1$;
\item there are applications of both types in the system, not all
  servers are occupied, the buffer is empty ---
  $\left\{(i_s; 0),(j_s; 0), R_1(i_s;j_s),1,1\right\}$,
  $\left\{(i_s; 0),(j_s; 0), R_1(i_s;j_s),2,1\right\}$,
  $\{(i_s; 0),(j_s; 0), R_2,1,2\}$, $\{(i_s; 0),(j_s; 0), R_2,2,2\}$,
  $1\leqslant  i_s \leqslant  n-2$, $1\leqslant  j_s \leqslant  n-1-i_s$;
\item there are applications of both types in the system, all servers
  are occupied, the buffer is empty ---
  $\left\{(i_s; 0),(n-i_s; 0), R_1(i_s;n-i_s),1,1\right\}$,
  $\{(i_s; 0),(n-i_s; 0), \\* R_1(i_s;n-i_s),2,1\}$,
  $\{(i_s; 0),n-i_s; 0), R_2,1,2\}$,
  $\{(i_s; 0),n-i_s; 0), R_2,2,2\}$, $1\leqslant  i_s \leqslant  n-1$;
\item there are applications of both types in the system, all servers
  are occupied, the buffer is not empty ---
  $\left\{(i_s; i_q),(n-i_s; j_q), R_1(i_s+i_q;n-i_s+
    j_q),1,1\right\}$,
  $\{(i_s; i_q), \\* (n-i_s; j_q), R_1(i_s+i_q; n-i_s+ j_q),2,1\}$,
  $\{(i_s; i_q),(n-i_s; j_q ), R_2,1,2\}$,
  $\{(i_s; i_q),(n-i_s; j_q), R_2,2,2\}$, $1\leqslant  i_s \leqslant  n-1$,
  $i_q+j_q\geqslant  1$.
\end{enumerate}

For the system with the buffer of finite size, three more groups of
states will be introduced (the system is fully occupied by
applications of only one type, the system is fully occupied by both
type applications).

Since the states in which the amount of resources requested by
applications exceeds the amount of resources of the entire system are
impossible (due to our assumptions), then then conditional indicator
function --- the Kronecker symbol --- is introduced:
\begin{equation}
  \delta\left(R_1(i_s+i_q,j_s+j_q)\right)=
  \left\{
    \begin{aligned}
      &1,\quad R_1-(i_s+i_q)k_1-(j_s+j_q)k_2\geqslant  0,\\
      &0, \quad R_1-(i_s+i_q)k_1-(j_s+j_q)k_2 < 0.
    \end{aligned}
  \right.
\end{equation}

This indicator function will be used for the equations of transitions
between the states of the groups (4), (6), (10) and for the transition
from the states (3), (6) and (9) to the overlying states and for
transitions from the overlying states to states of these groups.


The first four equations consider the transition of the system from
the zero state.
\begin{multline}
  \left(\lambda_1+\mu_{1,2}+\lambda_{1,2}\right)
  P\left((0;0),(0;0),R_1(0;0),1,1\right)=\mu_1
  P\left((1;0),(0;0),R_1(1;0),1,1\right)+\\ +\mu_1
  P\left((0;0),(1;0),R_1(0;1),1,1\right)+\lambda_{2,1}
  P\left((0;0),(0;0),R_1(0;0),2,1\right)+\\ +\mu_{2,1}
  P\left((0;0),(0;0),R_2,1,2\right),
\label{eq:1.1}
\end{multline}
%%%%%%%%%%%
\begin{multline}
  \left(\lambda_2+\mu_{1,2}+\lambda_{2,1}\right)
  P\left((0;0),(0;0),R_1(0;0),2,1\right)=\mu_1
  P\left((1;0),(0;0),R_1(1;0),2,1\right)+\\ +\mu_1
  P\left((0;0),(1;0),R_1(0;1),2,1\right)+\lambda_{1,2}
  P\left((0;0),(0;0),R_1(0;0),1,1\right)+\\ +\mu_{2,1}
  P\left((0;0),(0;0),R_2,2,2\right),
\label{eq:1.2}
\end{multline}
%%%%%%%%%%%%
\begin{multline}
  \left(\lambda_1+\mu_{2,1}+\lambda_{1,2}\right)
  P\left((0;0),(0;0),R_2,1,2\right)=\mu_2
  P\left((1;0),(0;0),R_2,1,2\right)+\\ +\mu_2
  P\left((0;0),(1;0),R_2,1,2\right)+\lambda_{2,1}
  P\left((0;0),(0;0),R_2,2,2\right)+\\ +\mu_{1,2}
  P\left((0;0),(0;0),R_1(0;0),1,1\right),
\label{eq:1.3}
\end{multline}
%%%%%%%%%%%
\begin{multline}
  \left(\lambda_2+\mu_{2,1}+\lambda_{2,1}\right)
  P\left((0;0),(0;0)R_2,2,2\right)=\mu_2
  P\left((1;0),(0;0),R_2,2,2\right)+\\ +\mu_2
  P\left((0;0),(1;0),R_2,2,2\right)+\lambda_{1,2}P\left((0;0),(0;0),R_2,1,2\right)+\\
  +\mu_{1,2}P\left((0;0),(0;0),R_1(0;0),2,1\right).
\label{eq:1.4}
\end{multline}

%%%%%%%%%%%%%%%%%%%%%%%%%%%%%%%%%%%%%%%%%%%%%
Now consider the case where only the first type of application is
present in the system and not all servers are occupied.
\begin{multline}
  \left(\lambda_1+\lambda_{1,2}+\mu_{1,2}+i_s\mu_1\right)
  P\left((i_s;0),(0;0),R_1(i_s;0),1,1\right)=\\ = \lambda_1
  P\left((i_s-1;0),(0;0),R_1(i_s-1;0),1,1\right) + \lambda_{2,1}
  P\left((i_s;0,)(0;0),R_1(i_s;0),2,1\right) + \\ + \mu_{2,1}
  P\left((i_s;0),(0;0),R_2,1,2\right)+\mu_1 (i_s+1)
  P\left((i_s+1;0),(0;0),R_1(i_s+1;0),1,1\right) + \\ + \mu_1
  P\left((i_s;0),(1;0),R_1(i_s;1),1,1\right),\quad 1\leqslant
  i_s\leqslant n-1,
\label{eq:2.1}
\end{multline}
%%%%%%
\begin{multline}
  \left(\lambda_2+\lambda_{2,1}+\mu_{1,2}+i_s\mu_1\right)
  P\left((i_s;0),(0;0),R_1(i_s;0),2,1\right)=\\ =
  \mu_1 (i_s+1) P\left((i_s+1;0),(0;0),R_1(i_s+1;0),2,1\right) + \\ +\mu_1
  P\left((i_s;0),(1;0),R_1(i_s;1),2,1\right)+ %\\ +
  \lambda_{1,2}
  P\left((i_s;0),(0;0),R_1(i_s;0),1,1\right) + \\ +\mu_{2,1}
  P\left((i_s;0),(0;0),R_2,2,2\right), \quad 1\leqslant i_s\leqslant
  n-1,
\label{eq:2.2}
\end{multline}
%%%%%
\begin{multline}
  \left(\lambda_1+\lambda_{1,2}+\mu_{2,1}+i_s\mu_2\right)
  P\left((i_s;0),(0;0),R_2,1,2\right)= \\ = \lambda_1
  P\left((i_s-1;0),(0;0),R_2,1,2\right) + %\\ +
  \lambda_{2,1}  P\left((i_s;0),(0;0),R_2,2,2\right) + \\ + \mu_{1,2}
  P\left((i_s;0),(0;0),R_1(i_s;0),1,1\right) + %\\ 
  \mu_2 (i_s+1) P\left((i_s+1;0),(0;0),R_2,1,2\right) + \\ + \mu_2
  P\left((i_s;0),(1;0),R_2,1,2\right),\quad 1\leqslant i_s\leqslant
  n-1,
\label{eq:2.3}
\end{multline}
%%%%%
\begin{multline}
  \left(\lambda_2+\lambda_{2,1}+\mu_{2,1}+i_s\mu_2\right)
  P\left((i_s;0),(0;0),R_2,2,2\right)= \lambda_{1,2}
  P\left((i_s;0),(0;0),R_2,1,2\right) + \\ + \mu_{1,2}
  P\left((i_s;0),(0;0),R_1(i_s;0),2,1\right)+ \mu_2 (i_s+1)
  P\left((i_s+1;0),(0;0),R_2,2,2\right) + \\ + \mu_2
  P\left((i_s;0),(1;0),R_2,2,2\right), \quad 1\leqslant i_s\leqslant
  n-1.
\label{eq:2.4}
\end{multline}
%%%%%%%%%%%%%%%%%%%%%%%%%%%%%%%%%%%%%%%%%

The system contains only applications of the first type, all servers
are occupied, but the buffer is empty. According to the assumptions,
the maximum amount of system resources is sufficient for all servers
to be occupied, but it is not sufficient for arriving applications to
occupy the buffer. Therefore, it is necessary to use the indicator
function --- verification of the existence of overlying states.
\begin{multline}
  \left(n\mu_1+\mu_{1,2}+\lambda_{1,2}+\lambda_1 \delta\left(R_1
      (n+1,0)\right)\right) P\left((n,0),(0,0),R_1 (n;0),1,1\right) =
  \\ = \lambda_1 P\left((n-1,0),(0,0),R_1
    (n-1;0),1,1\right)+\lambda_{2,1} P\left((n,0),(0,0),R_1
    (n;0),2,1\right)+ \\ + \mu_{2,1}
  P\left((n,0),(0,0),R_2,1,2\right)+\mu_1 \delta\left(R_1 (n;1)\right)
  P\left((n-1,1),(1,0),R_1 (n;1),1,1\right)+\\ + \mu_{2,1}
  \sum\limits_{i+j=1}^{\infty}\prod\limits_{i_1+j_1=1)}^{j+j}\left(1-\delta\left(R_1
      (n+i_1;j_1) \right)\right) P\left((n,i),(0,j),R_2,1,2\right)+ \\
  + n\mu_1 \delta\left(R_1 (n+1,0)\right) P\left((n,1),(0,0),R_1
    (n+1;0),1,1\right),
\label{eq:3.1}
\end{multline}
%%%%%%%%%%%
\begin{multline}
  \left(n\mu_1+\mu_{1,2}+\lambda_{2,1}+\lambda_2 \delta\left(R_1
      (n,1)\right)\right) P\left((n,0),(0,0),R_1 (n;0),2,1\right) = \\
  = \lambda_{1,2} P\left((n,0)(0,0),R_1 (n;0),1,1\right)+ \mu_1
  \delta\left(R_1 (n,1)\right) P\left((n-1,1),(1,0),R_1
    (n;1),2,1\right)+\\ + \mu_{2,1} P\left((n,0),(0,0),R_2,2,2\right)+
  n\mu_1 \delta\left(R_1 (n+1,0)\right) P\left((n,1),(0,0), R_1
    (n+1;0),2,1\right)+ \\ + \mu_{2,1}
  \sum\limits_{i+j=1}^{\infty}\prod\limits_{i_1+j_1=1)}^{j+j}\left(1-\delta\left(R_1
      (n+i_1;j_1) \right)\right) P\left((n,i),(0,j),R_2,2,2\right),
\label{eq:3.2}
\end{multline}
%%%%%%%%%%%
\begin{multline}
  \left(n\mu_2+\mu_{2,1}+\lambda_{1,2}+\lambda_1 \right)
  P\left((n,0),(0,0),R_2,1,2\right) =
  \lambda_1 P\left((n-1,0),(0,0),R_2,1,2\right)+\\
  \lambda_{2,1} P\left((n,0),(0,0),R_2,2,2\right)+ \mu_{1,2}
  P\left((n,0),(0,0), R_1(n;0),1,1\right)+\\ + \mu_2
  P\left((n-1,1),(1,0),R_2,1,2\right)+ n\mu_2 P\left((n,1),(0,0),
    R_2,1,2\right),
\label{eq:3.3}
\end{multline}
%%%%%%%%%%%
\begin{multline}
  \left(n\mu_2+\mu_{2,1}+\lambda_{2,1}+\lambda_2\right)
  P\left((n,0),(0,0),R_2,2,2\right) = \lambda_{1,2}
  P\left((n,0),(0,0),R_2,1,2\right)+ \\ + \mu_{1,2}
  P\left((n,0),(0,0),R_1(n;0),2,1\right)+\mu_2
  P\left((n-1,1),(1,0),R_2,2,2\right)+\\ + n\mu_2 P\left((n,1),(0,0),
    R_2,2,2\right),
\label{eq:3.4}
\end{multline}
%%%%%%%%%%%%%%%%%%%%%%%%%%%%%%%%%%%%%%%%%%%%%%

In the system there are only applications of the first type, all
servers are occupied, there are also applications in the buffer. The
indicator function is used to check the possibility of transition to
(from) overlying states.
\begin{multline}
  \left(n\mu_1+\mu_{1,2}+\lambda_{1,2}+\lambda_1 \delta\left(R_1
      (n+i_q+1,0)\right)\right) P\left((n,i_q),(0,0),R_1
    (n+i_q;0),1,1\right) = \\ = \lambda_1 P\left((n,i_q-1),(0,0),R_1
    (n+i_q-1;0),1,1\right)+ \\ +\lambda_{2,1} P\left((n,i_q),(0,0),R_1
    (n+i_q;0),2,1\right)+ % \\ +
  \mu_{2,1}
  P\left((n,i_q),(0,0),R_2,1,2\right)+ \\ +\mu_1 \delta\left(R_1
    (n+i_q;1)\right) P\left((n-1,i_q+1)(1,0),R_1 (n+i_q;1),1,1\right)+
  \\ + \mu_{2,1}
  \sum\limits_{i+j=1}^{\infty}\prod\limits_{i_1+j_1=1)}^{j+j}\left(1-\delta\left(R_1
      (n+i_q+i_1;j_1) \right)\right)
  P\left((n,i_q+i),(0,j),R_2,1,2\right)+\\ + n\mu_1 \delta\left(R_1
    (n+i_q+1,0)\right) P\left((n,i_q+1),(0,0),R_1
    (n+i_q+1;0),1,1\right), \quad i_q\geqslant 1,
\label{eq:4.1}
\end{multline}
%%%%%%%%
\begin{multline}
  \left(n\mu_1+\mu_{1,2}+\lambda_{2,1}+\lambda_2\delta\left(R_1
      (n+i_q+1,0)\right)\right) P\left((n,i_q)(0,0),R_1
    (n+i_q;0),2,1\right) = \\ =
%\lambda_1 P\left((n,i_q-1)(0,0),R_1 (n+i_q-1;0),1,1\right)+
  \lambda_{1,2} P\left((n,i_q),(0,0),R_1 (n+i_q;0),1,1\right)+
  \mu_{2,1} P\left((n,i_q),(0,0),R_2,2,2\right)+\\ + n\mu_1
  \delta\left(R_1 (n+i_q+1,0)\right) P\left((n,i_q+1),(0,0),R_1
    (n+i_q+1;0),2,1\right) +\\ + \mu_{2,1}
  \sum\limits_{i+j=1}^{\infty}\prod\limits_{i_1+j_1=1)}^{j+j}\left(1-\delta\left(R_1
      (n+i_q+i_1;j_1) \right)\right)
  P\left((n,i_q+i),(0,j),R_2,2,2\right)+\\ + \mu_1 \delta\left(R_1
    (n+i_q;1)\right) P\left((n-1;i_q+1),(1,0),R_1
    (n+i_q;1),2,1\right), \quad i_q\geqslant 1,
\label{eq:4.2}
\end{multline}
%%%%%%%%%
\begin{multline}
  \left(n\mu_2+\mu_{2,1}+\lambda_{1,2}+\lambda_1\right)
  P\left((n,i_q),(0,0),R_2,1,2\right) = \\ =\lambda_1
  P\left((n,i_q-1),(0,0),R_2,1,2\right)+ \lambda_{1,2}
  P\left((n,i_q),(0,0),R_2,1,2\right)+ \\+ 
  \mu_{1,2}\delta\left(R_1(n+i_q;0)\right)
  P\left((n,i_q),(0,0),(R_1(n+i_q;0),1,1\right)+\\ +
  \mu_2 P\left((n-1,i_q+1),(1,0),R_2,1,2\right)+ \\+ n\mu_2
  P\left((n,i_q+1),(0,0),R_2,1,2\right), \quad i_q\geqslant 1,
\label{eq:4.3}
\end{multline}
%%%%%%%
\begin{multline}
  \left(n\mu_2+\mu_{2,1}+\lambda_{2,1}+\lambda_2\right)
  P\left((n,i_q),(0,0),R_2,2,2\right) =  \lambda_{2,1}
  P\left((n,i_q),(0,0),R_2,2,2\right)+ \\ +
  \mu_{1,2}\delta\left(R_1(n+i_q;0)\right)
  P\left((n,i_q),(0,0),(R_1(n+i_q;0),2,1\right)+\\ +
  \mu_2 P\left((n-1,i_q+1),(1,0),R_2,2,2\right)+ n\mu_2
  P\left((n,i_q+1),(0,0),R_2,2,2\right), \ i_q\geqslant 1,
\label{eq:4.4}
\end{multline}
%%%%%%%%%%%%%%%%%%%%%%%%%%%%%%%%%%%%%%%%%%%%%%%%

There are only application of the second type in the system, not all
servers are occupied, the buffer is empty.
\begin{multline}
  \left(\lambda_1+\lambda_{1,2}+\mu_{1,2}+j_s\mu_1\right)
  P\left((0;0),(j_s;0),R_1(0;j_s),1,1\right)=\\ = \mu_1 (j_s+1)
  P\left((0;0),(j_s+1;0),R_1(0;j_s+1),1,1\right) + \\ + \mu_1
  P\left((1;0),(j_s;0),R_1(1;j_s),1,1\right) + \lambda_{2,1}
  P\left((0;0)(j_s;0),R_1(0;j_s),2,1\right) + \\ +\mu_{2,1}
  P\left((0;0),(j_s;0),R_2,1,2\right), \quad 1\leqslant j_s\leqslant
  n-1,
\label{eq:5.1}
\end{multline}
%%%%%%
\begin{multline}
  \left(\lambda_2+\lambda_{2,1}+\mu_{1,2}+j_s\mu_1\right)
  P\left((0;0),(j_s;0),R_1(0;j_s),2,1\right)=\\ = \lambda_2
  P\left((0;0),(j_s-1;0),R_1(0;j_s-1),2,1\right) + \mu_1
  P\left((1;0),(j_s;0),R_1(1;j_s),2,1\right)+\\ + \mu_1 (j_s+1)
  P\left((0;0),(j_s+1;0),R_1(0;j_s+1),2,1\right) +\\ + \lambda_{1,2}
  P\left((0;0),(j_s;0),R_1(0;j_s),1,1\right) + \\ + \mu_{2,1}
  P\left((0;0),(j_s;0),R_2,2,2\right), \quad 1\leqslant j_s\leqslant
  n-1,
\label{eq:5.2}
\end{multline}
%%%%%
\begin{multline}
  \left(\lambda_1+\lambda_{1,2}+\mu_{2,1}+j_s\mu_2\right)
  P\left((0;0),(j_s;0),R_2,1,2\right)= \\ = \lambda_{2,1}
  P\left((0;0),(j_s;0),R_2,2,2\right) + \mu_{1,2}
  P\left((0;0),(j_s;0),R_1(0;j_s),1,1\right)+\\ + \mu_2 (j_s+1)
  P\left((0;0),(i_s+1;0),R_2,1,2\right) + \\ +\mu_2
  P\left((1;0),(i_s;0),R_2,1,2\right),\quad 1\leqslant j_s\leqslant
  n-1,
\label{eq:5.3}
\end{multline}
%%%%%
\begin{multline}
  \left(\lambda_2+\lambda_{2,1}+\mu_{2,1}+j_s\mu_2\right)
  P\left((0;0),(j_s;0),R_2,2,2\right)=\\ = \lambda_2
  P\left((0;0),(j_s-1;0),R_2,2,2\right) + \lambda_{1,2}
  P\left((0;0),(j_s;0),R_2,1,2\right) + \\ + \mu_{1,2}
  P\left((0;0),(j_s;0),R_1(0;j_s),2,1\right)+ \mu_2 (j_s+1)
  P\left((0;0),(j_s+1;0),R_2,2,2\right) + \\ + \mu_2
  P\left((1;0),(j_s;0),R_2,2,2\right), \quad 1\leqslant i_s\leqslant
  n-1,
\label{eq:5.4}
\end{multline}
%%%%%%%%%%%%%%%%%%%%%%%%%%%%%%%%%%%%%%%%%

In the system there are only applications of the second type, all
servers are occupied, but the buffer is empty. The indicator function
is used to check the possibility of transition to (from) overlying
states.
\begin{multline}
  \left(n\mu_1+\mu_{1,2}+\lambda_{1,2}+\lambda_1 \delta\left(R_1
      (1,n)\right)\right) P\left((0,0),(n,0),R_1 (0;n),1,1\right) = \\
  = \lambda_{2,1} P\left((0,0),(n,0),R_1 (0;n),2,1\right)+ \\ +
  \mu_{2,1} P\left((0,0),(n,0),R_2,1,2\right)+\mu_1 \delta\left(R_1
    (1;n)\right) P\left(1,0),((n-1,1),R_1 (1;n),1,1\right)+\\ +
  \mu_{2,1}
  \sum\limits_{i+j=1}^{\infty}\prod\limits_{i_1+j_1=1)}^{j+j}\left(1-\delta\left(R_1
      (i_1;n+j_1) \right)\right) P\left((0,i),(n,j),R_2,1,2\right)+ \\
  + n\mu_1 \delta\left(R_1 (0,n+1)\right) P\left((0;0),(n;1),R_1
    (0;n+1),1,1\right),
\label{eq:6.1}
\end{multline}
%%%%%%%%%%%
\begin{multline}
  \left(n\mu_1+\mu_{1,2}+\lambda_{2,1}+\lambda_2 \delta\left(R_1
      (0,n+1)\right)\right) P\left((0,0),(n,0),R_1 (0;n),2,1\right) =
  \\ = \lambda_2 P\left((0,0),(n-1,0),R_1 (0;n-1),1,1\right)+
  \lambda_{1,2} P\left((0,0)(n,0),R_1 (0;n),1,1\right)+ \\ + \mu_1
  \delta\left(R_1 (1,n)\right) P\left(1,0),((n-1,1),R_1
    (1;n),2,1\right)+\mu_{2,1} P\left((0,0),(n,0),R_2,2,2\right)+\\ +
  n\mu_1 \delta\left(R_1 (0,n+1)\right) P\left((0,0),(n,1), R_1
    (0;n+1),2,1\right)+ \\ + \mu_{2,1}
  \sum\limits_{i+j=1}^{\infty}\prod\limits_{i_1+j_1=1)}^{j+j}\left(1-\delta\left(R_1
      (i_1;n+j_1) \right)\right) P\left((0,i),(n,j),R_2,2,2\right),
\label{eq:6.2}
\end{multline}
%%%%%%%%%%%
\begin{multline}
  \left(n\mu_2+\mu_{2,1}+\lambda_{1,2}+\lambda_1 \right)
  P\left((0,0),(n,0),R_2,1,2\right) = \lambda_{2,1}
  P\left((0,0),(n,0),R_2,2,2\right)+ \\ +
  \mu_{1,2} P\left((0,0),(n,0), R_1(0;n),1,1\right)+ \mu_2
  P\left((1,0),(n-1,1),R_2,1,2\right)+ \\ +
  n\mu_2 P\left((0,0),(n,1), R_2,1,2\right),
\label{eq:6.3}
\end{multline}
%%%%%%%%%%%
\begin{multline}
  \left(n\mu_2+\mu_{2,1}+\lambda_{2,1}+\lambda_2\right)
  P\left((0,0),(n,0),R_2,2,2\right) = \lambda_2
  P\left((0,0),(n-1,0),R_2,1,2\right)+\\ + \lambda_{1,2}
  P\left((0,0),(n,0),R_2,1,2\right)+ \mu_{1,2}
  P\left((0,0),(n,0),R_1(0;n),2,1\right)+\\ + \mu_2
  P\left((1,0),(n-1,1),R_2,2,2\right)+ n\mu_2 P\left((0,0),(n,1),
    R_2,2,2\right),
\label{eq:6.4}
\end{multline}
%%%%%%%%%%%%%%%%%%%%%%%%%%%%%%%%%%%%%%%%%%%%%%

In the system there are only applications of the second type, all
servers are occupied, the buffer is not empty. The indicator function
is used to check the possibility of transition to (from) overlying
states.
\begin{multline}
  \left(n\mu_1+\mu_{1,2}+\lambda_{1,2}+\lambda_1 \delta\left(R_1
      (1,n+j_q)\right)\right) P\left((0,0),(n,j_q),R_1
    (0;n+j_q),1,1\right) = \\ = \lambda_{2,1} P\left((0,0),(n,j_q),R_1
    (0;n+j_q),2,1\right)+ \mu_{2,1}
  P\left((0,0,(n,j_q),R_2,1,2\right)+ \\ + \mu_1 \delta\left(R_1
    (1;n+J_q)\right) P\left((1,0)(n-1,j_q+1),R_1 (1;n+j_q),1,1\right)+
  \\ + \mu_{2,1}
  \sum\limits_{i+j=1}^{\infty}\prod\limits_{i_1+j_1=1)}^{j+j}\left(1-\delta\left(R_1
      (i_1;n+j_q+j_1) \right)\right)
  P\left((0,i),(n,j_q+j),R_2,1,2\right)+\\ + n\mu_1 \delta\left(R_1
    (0, n+j_q+1)\right) P\left((0,0),(n,j_q+1),R_1
    (0;n+j_q+1),1,1\right), \quad j_q\geqslant 1,
\label{eq:7.1}
\end{multline}
%%%%%%%%
\begin{multline}
  \left(n\mu_1+\mu_{1,2}+\lambda_{2,1}+\lambda_2\delta\left(R_1
      (0;n+j_q+1)\right)\right) P\left((0,0)(n,j_q),R_1
    (0;n+j_q),2,1\right) = \\ = \lambda_2 P\left((0,0),(n,j_q-1),R_1
    (0;n+j_q-1),1,1\right)+ \\ + \lambda_{1,2} P\left((0,0),(n,j_q),R_1
    (0;n+j_q),1,1\right)+ % \\ +
  \mu_{2,1}
  P\left((0,0),(n,j_q),R_2,2,2\right)+\\ + n\mu_1 \delta\left(R_1
    (0,n+j_q+1)\right) P\left((0,0),(n,j_q+1),R_1
    (0;n+j_q+1),2,1\right) +\\ + \mu_{2,1}
  \sum\limits_{i+j=1}^{\infty}\prod\limits_{i_1+j_1=1)}^{j+j}\left(1-\delta\left(R_1
      (i_1;n+i_q+j_1) \right)\right)
  P\left((0,i),(n,j_q+j),R_2,2,2\right)+\\ + \mu_1 \delta\left(R_1
    (1;n+j_q)\right) P\left((1,0),(n-1;j_q+1),R_1
    (1;n+j_q),2,1\right), \quad j_q\geqslant 1,
\label{eq:7.2}
\end{multline}
%%%%%%%%%
\begin{multline}
  \left(n\mu_2+\mu_{2,1}+\lambda_{1,2}+\lambda_1\right)
  P\left((0,0),(n,j_q),R_2,1,2\right) =
  \lambda_{1,2} P\left((0,0),(n,j_q),R_2,1,2\right)+ \\ +
  \mu_{1,2}\delta\left(R_1(0;n+j_q)\right)
  P\left((0,0),(n,j_q),(R_1(0;n+j_q),1,1\right)+\\ +
  \mu_2 P\left((1,0),(n-1,j_q+1),R_2,1,2\right)+ \\ + n\mu_2
  P\left((0,0),(n,j_q+1),R_2,1,2\right), \quad j_q\geqslant  1,
\label{eq:7.3}
\end{multline}
%%%%%%%
\begin{multline}
  \left(n\mu_2+\mu_{2,1}+\lambda_{2,1}+\lambda_2\right)
  P\left((0,0),(n,j_q),R_2,2,2\right) =
  \lambda_2 P\left((0,0),(n,j_q-1),R_2,1,2\right)+ \\ +
  \mu_2  P\left((1,0),(n-1,j_q+1),R_2,2,2\right)+ \lambda_{2,1}
  P\left(0,0),((n,j_q),R_2,2,2\right)+ \\ +
  \mu_{1,2}\delta\left(R_1(0;n+j_q)\right)
  P\left((0,0),(n,j_q),(R_1(0;n+j_q),2,1\right)+\\ +
  n\mu_2 P\left((0,0),(n,j_q+1),R_2,2,2\right), \quad j_q\geqslant  1,
\label{eq:7.4}
\end{multline}
%%%%%%%%%%%%%%%%%%%%%%%%%%%%%%%%%%%%%%%%%%%%%%%%

The applications of both types are in the system, but only some (not
all) servers are occupied.
\begin{multline}
  \left(\lambda_1+\mu_{1,2}+\lambda_{1,2}+(i_s+j_s)\mu_1\right)
  P\left((i_s,0),(j_s,0),R_1(i_s;j_s),1,1\right)=\\ =
  \lambda_1
  P\left((i_s-1,0),(j_s,0),R_1(i_s-1;j_s),1,1\right)+\lambda_{2,1}
  P\left((i_s,0),(j_s,0),R_1(i_s;j_s),2,1\right)+\\ +
  \mu_{2,1} P\left((i_s,0),(j_s,0),R_2,1,2\right)+(i_s+1)\mu_1
  P\left((i_s+1,0),(j_s,0),R_1(i_s+1;j_s),1,1\right)+\\ +
  (j_s+1)\mu_1 P\left((i_s,0),(j_s+1,0),R_1(i_s;j_s+1),1,1\right), \\
  \quad i_s=\overline{1,n-2}, \quad j_s=\overline{1,n-1-i_s},
\label{eq:8.1}
\end{multline}
%%%%%%%%%%%
\begin{multline}
  \left(\lambda_2+\mu_{1,2}+\lambda_{2,1}+(i_s+j_s)\mu_1\right)
  P\left((i_s,0),(j_s,0),R_1(i_s;j_s),2,1\right)=\\ =
  \lambda_2
  P\left((i_s,0),(j_s-1,0),R_1(i_s;j_s-1),2,1\right)+\lambda_{1,2}
  P\left((i_s,0),(j_s,0),R_1(i_s;j_s),1,1\right)+\\ +
  \mu_{2,1} P\left((i_s,0),(j_s,0),R_2,2,2\right)+(i_s+1)\mu_1
  P\left((i_s+1,0),(j_s,0),R_1(i_s+1;j_s),2,1\right)+\\ +
  (j_s+1)\mu_1 P\left((i_s,0),(j_s+1,0),R_1(i_s;j_s+1),2,1\right), \\
  \quad i_s=\overline{1,n-2}, \quad j_s=\overline{1,n-1-i_s},
\label{eq:8.2}
\end{multline}
%%%%%%%%%%%
\begin{multline}
  \left(\lambda_1+\mu_{2,1}+\lambda_{1,2}+(i_s+j_s)\mu_2\right)
  P\left((i_s,0),(j_s,0),R_2,1,2\right)=\\ = \lambda_1
  P\left((i_s-1,0),(j_s,0),R_2,1,2\right)+\lambda_{2,1}
  P\left((i_s,0),(j_s,0),R_2,2,2\right)+\\ + \mu_{1,2}
  P\left((i_s,0),(j_s,0),R_1(i_s;j_s),1,1\right)+(i_s+1)\mu_2
  P\left((i_s+1,0),(j_s,0),R_2,1,2\right)+\\ + (j_s+1)\mu_2
  P\left((i_s,0),(j_s+1,0),R_2,1,2\right), \quad i_s=\overline{1,n-2},
  \quad j_s=\overline{1,n-1-i_s},
\label{eq:8.3}
\end{multline}
%%%%%%%%
\begin{multline}
  \left(\lambda_2+\mu_{2,1}+\lambda_{2,1}+(i_s+j_s)\mu_2\right)
  P\left((i_s,0),(j_s,0),R_2,2,2\right)=\\ =
  \lambda_2 P\left((i_s,0),(j_s-1,0),R_2,2,2\right)+\lambda_{1,2}
  P\left((i_s,0),(j_s,0),R_2,1,2\right)+\\ +
  \mu_{1,2} P\left((i_s,0),(j_s,0)R_1(i_s;j_s),2,1\right)+(i_s+1)\mu_2
  P\left((i_s+1,0),(j_s,0),R_2,2,2\right)+\\ +
  (j_s+1)\mu_2 P\left((i_s,0),(j_s+1,0),R_2,2,2\right), \quad
  i_s=\overline{1,n-2}, \quad j_s=\overline{1,n-1-i_s},
\label{eq:8.4}
\end{multline}
%%%%%%%%%%%%%%%%%%%%%%%%%%%%%%%%%%%%%%

The application of the first and the second types are in the system,
all servers are occupied, but the buffer is empty.
\begin{multline}
  \left(n\mu_1+\mu_{1,2}+\lambda_{1,2}+\lambda_1 \delta\left(R_1
      (i_s+1,n-i_s)\right)\right) P\left((i_s,0),(n-i_s,0),R_1
    (i_s;n-i_s),1,1\right) = \\ = \lambda_1
  P\left((i_s-1,0),(n-i_s,0),R_1 (i_s-1;n-i_s),1,1\right)+\\ +
  \lambda_{2,1} P\left((i_s,0),(n-i_s,0),R_1 (i_s;n-i_s),2,1\right)+
  \mu_{2,1} P\left((i_s,0),(n-i_s,0),R_2,1,2\right)+ \\ +
  i_s\mu_1 \delta\left(R_1 (i_s+1;n-i_s)\right)
  P\left((i_s,1),(n-i_s,0),R_1 (i_s+1;n-i_s),1,1\right)+\\ +
  (n-i_s)\mu_1 \delta\left(R_1 (i_s;n-i_s+1)\right)
  P\left((i_s,0),(n-i_s,1),R_1 (i_s;n-i_s+1),1,1\right)+\\ +
  (n-i_s+1)\mu_1 \delta\left(R_1 (i_s;n-i_s+1)\right)
  P\left((i_s-1,1),(n-i_s+1,0),R_1 (i_s;n-i_s+1),1,1\right)+\\ +
  \mu_{2,1} \sum\limits_{i+j=1}^{\infty}\prod\limits_{i_1+j_1=1)}^{j+j}\left(1-\delta\left(R_1 (i_s+i_1;n-i_s+j_1) \right)\right) P\left((i_s,i),(n-i_s,j),R_2,1,2\right), \\
  \quad i_s=\overline{1,n-1}, \quad j_s=n-i_s,
\label{eq:9.1}
\end{multline}
%%%%%%%
\begin{multline}
  \left(n\mu_1+\mu_{1,2}+\lambda_{2,1}+\lambda_2 \delta\left(R_1
      (i_s,n-i_s+1)\right)\right) P\left((i_s,0),(n-i_s,0),R_1
    (i_s;n-i_s),2,1\right) = \\ = \lambda_2
  P\left((i_s,0),(n-i_s-1,0),R_1 (i_s;n-i_s-1),2,1\right)+\\ +
  \lambda_{1,2} P\left((i_s,0),(n-i_s,0),R_1 (i_s;n-i_s),1,1\right)+
  \mu_{2,1} P\left((i_s,0),(n-i_s,0),R_2,2,2\right)+ \\ + i_s\mu_1
  \delta\left(R_1 (i_s+1;n-i_s)\right) P\left((i_s,1),(n-i_s,0),R_1
    (i_s+1;n-i_s),2,1\right)+\\ + (n-i_s+1)\mu_1 \delta\left(R_1
    (i_s;n-i_s+1)\right) P\left((i_s-1,1),(n-i_s+1,0),R_1
    (i_s;n-i_s+1),2,1\right)+\\ + (n-i_s)\mu_1 \delta\left(R_1
    (i_s;n-i_s+1)\right) P\left((i_s,0),(n-i_s,1),R_1
    (i_s;n-i_s+1),2,1\right)+\\ + \mu_{2,1}
  \sum\limits_{i+j=1}^{\infty}\prod\limits_{i_1+j_1=1}^{j+j}\left(1-\delta\left(R_1
      (i_s+i_1;n-i_s+j_1) \right)\right)
  P\left((i_s,i),(n-i_s,j),R_2,2,2\right), \\
  \quad i_s=\overline{1,n-1}, \quad j_s=n-i_s,
\label{eq:9.2}
\end{multline}
%%%%%%%
\begin{multline}
  \left(n\mu_2+\mu_{2,1}+\lambda_{1,2}+\lambda_1 \right)
  P\left((i_s,0),(n-i_s,0),R_2,1,2\right) = \\ =
  \lambda_1 P\left((i_s-1,0),(n-i_s,0),R_2,1,2\right)+ \lambda_{2,1}
  P\left((i_s,0),(n-i_s,0),R_2,2,2\right)+ \\ +
  \mu_{1,2} P\left((i_s,0),(n-i_s,0),R_1(i_s;j_s),1,1\right)+
  (n-i_s)\mu_2 P\left((i_s,0),(n-i_s,1),R_2,1,2\right) +\\ +
  i_s\mu_2 P\left((i_s,1),(n-i_s,0),R_2,1,2\right) + (n-i_s+1)\mu_2  P\left((i_s-1,1),(n-i_s+1,0),R_2,1,1\right), \\
  \quad i_s=\overline{1,n-1}, \quad j_s=n-i_s,
\label{eq:9.3}
\end{multline}
%%%%%%%%
\begin{multline}
  \left(n\mu_2+\mu_{2,1}+\lambda_{2,1}+\lambda_2 \right)
  P\left((i_s,0),(n-i_s,0),R_2,2,2\right) = \\ = \lambda_2
  P\left((i_s,0),(n-i_s-1,0),R_2,2,2\right)+ \lambda_{1,2}
  P\left((i_s,0),(n-i_s,0),R_2,1,2\right)+ \\ + \mu_{1,2}
  P\left((i_s,0),(n-i_s,0),R_1(i_s;j_s),2,1\right)+ (n-i_s)\mu_2
  P\left((i_s,0),(n-i_s,1),R_2,2,2\right)+\\ +
  (n-i_s+1)\mu_2 P\left((i_s-1,1),(n-i_s+1,0),R_2,2,2\right)+  i_s\mu_2  P\left((i_s,1),(n-i_s,0),R_2,2,2\right), \\
  \quad i_s=\overline{1,n-1}, \quad j_s=n-i_s,
\label{eq:9.4}
\end{multline}
%%%%%%%%%%%%%%%%%%%%%%%%%%%%%%%%%%%%%%

The equations for the case when both types of applications are in the
system (on servers and in the buffer).
\begin{multline}
  \left(\lambda_1 \delta\left(R_1(i_s+i_q+1,n-i_s+j_q)\right)+\lambda_{1,2}+\mu_{1,2}+n\mu_1\right) \times \\
  \times P\left((i_s;i_q),(n-i_s;j_q),R_1(i_s+i_q;
    n-i_s+j_q),1,1\right) = \\ = \lambda_1
  P\left((i_s;i_q-1),(n-i_s;j_q),R_1(i_s+i_q-1;n-i_s+j_q),1,1\right) +
  \\ + \lambda_{2,1} P\left((i_s;i_q),(n-i_s;j_q),R_1(i_s+i_q;
    n-i_s+j_q),2,1\right) + \\ + \mu_{2,1}
  P\left((i_s;i_q),(n-i_s;j_q),R_2,1,2\right) + \\ +
  \mu_{2,1} \sum\limits_{i+j=1}^{\infty}\prod\limits_{i_1+j_1=1}^{j+j}\left(1-\delta\left(R_1 (i_s+i_q+i_1;n-i_s+j_q+j_1) \right)\right) \times \\
  \times P\left((i_s,i_q+i),(n-i_s,j_q+j),R_2,1,2\right) +
  i_s p_1\mu_1\delta\left(R_1(i_s+i_q+1;n-i_s+j_q)\right) \times \\
  \times P\left((i_s;i_q+1),(n-i_s;j_q),R_1(i_s+i_q+1;
    n-i_s+j_q),1,1\right) + \\ +
  (n-i_s) p_2 \mu_1 \delta\left(R_1(i_s+i_q;n-i_s+j_q+1)\right) \times \\
  \times P\left((i_s;i_q),(n-i_s;j_q+1),R_1(i_s+i_q;
    n-i_s+j_q+1),1,1\right) + \\ +
  (i_s+1) p_2 \mu_1 \delta\left(R_1(i_s+i_q+1;n-i_s+j_q)\right)  \times \\
  \times P\left((i_s+1;i_q),(n-i_s-1;j_q+1),R_1(i_s+i_q+1;
    n-i_s+j_q),1,1\right) + \\ +
  (n-i_s+1) p_1\mu_1 \delta\left(R_1(i_s+i_q;n-i_s+j_q+1)\right) \times \\
  \times  P\left((i_s-1;i_q+1),(n-i_s+1;j_q),R_1(i_s+i_q; n-i_s+j_q+1),1,1\right), \\
  i_s=\overline{1,n-1}, \quad i_q+j_q\geqslant 1,
\label{eq:10.1}
\end{multline}
%%%%%%%%%%%%%%%%%%%%%%%%%%%%%%%%%%
\begin{multline}
  \left(\lambda_2 \delta\left(R_1(i_s+i_q,n-i_s+j_q+1)\right)+\lambda_{2,1}+\mu_{1,2}+n\mu_1\right) \times \\
  \times  P\left((i_s;i_q),(n-i_s;j_q),R_1(i_s+i_q;
    n-i_s+j_q),2,1\right) = \\ =
  \lambda_2
  P\left((i_s;i_q),(n-i_s;j_q-1),R_1(i_s+i_q;n-i_s+j_q-1),2,1\right) +
  \\ +
  \lambda_{1,2} P\left((i_s;i_q),(n-i_s;j_q),R_1(i_s+i_q;
    n-i_s+j_q),1,1\right) + \\ +
  \mu_{2,1}  P\left((i_s;i_q),(n-i_s;j_q),R_2,2,2\right) + \\ +
  \mu_{2,1} \sum\limits_{i+j=1}^{\infty}\prod\limits_{i_1+j_1=1}^{j+j}\left(1-\delta\left(R_1 (i_s+i_q+i_1;n-i_s+j_q+j_1) \right)\right) \times \\
  \times P\left((i_s,i_q+i),(n-i_s,j_q+j),R_2,2,2\right) +
  i_s p_1\mu_1\delta\left(R_1(i_s+i_q+1;n-i_s+j_q)\right) \times \\
  \times P\left((i_s;i_q+1),(n-i_s;j_q),R_1(i_s+i_q+1;
    n-i_s+j_q),2,1\right) + \\ +
  (n-i_s) p_2 \mu_1 \delta\left(R_1(i_s+i_q;n-i_s+j_q+1)\right) \times \\
  \times P\left((i_s;i_q),(n-i_s;j_q+1),R_1(i_s+i_q;
    n-i_s+j_q+1),2,1\right) + \\ +
  (i_s+1) p_2 \mu_1 \delta\left(R_1(i_s+i_q+1;n-i_s+j_q)\right)  \times \\
  \times P\left((i_s+1;i_q),(n-i_s-1;j_q+1),R_1(i_s+i_q+1;
    n-i_s+j_q),2,1\right) + \\ +
  (n-i_s+1) p_1\mu_1 \delta\left(R_1(i_s+i_q;n-i_s+j_q+1)\right) \times \\
  \times  P\left((i_s-1;i_q+1),(n-i_s+1;j_q),R_1(i_s+i_q; n-i_s+j_q+1),2,1\right), \\
  i_s=\overline{1,n-1}, \quad i_q+j_q\geqslant  1,
\label{eq:10.2}
\end{multline}
%%%%%%%
\begin{multline}
  \left(\lambda_1 +\lambda_{1,2}+\mu_{1,2}+n\mu_2\right)
  P\left((i_s;i_q),(n-i_s;j_q),R_2,1,2\right) = \\ =
  \lambda_1 P\left((i_s;i_q-1),(n-i_s;j_q),R_2,1,2\right) +
  \lambda_{2,1} P\left((i_s;i_q),(n-i_s;j_q),R_2,2,2\right) + \\ +
  \mu_{1,2}  \delta\left(R_1(i_s+i_q; n-i_s+j_q)\right)
  P\left((i_s;i_q),(n-i_s;j_q),R_1(i_s+i_q; n-i_s+j_q),1,1\right) + \\ +
  i_s p_1\mu_2  P\left((i_s;i_q+1),(n-i_s;j_q),R_2,1,2\right) + \\ +
  (n-i_s) p_2 \mu_2  P\left((i_s;i_q),(n-i_s;j_q+1),R_2,1,2\right) +
  \\ +
  (i_s+1) p_2 \mu_2  P\left((i_s+1;i_q),(n-i_s-1;j_q+1),R_2,1,2\right)
  + \\ +
  (n-i_s+1) p_1\mu_2 P\left((i_s-1;i_q+1),(n-i_s+1;j_q),R_2,1,2\right), \\
  i_s=\overline{1,n-1}, \quad i_q+j_q\geqslant  1,
\label{eq:10.3}
\end{multline}
%%%%%%%%
\begin{multline}
  \left(\lambda_2 +\lambda_{2,1}+\mu_{1,2}+n\mu_2\right)
  P\left((i_s;i_q),(n-i_s;j_q),R_2,2,2\right) = \\ =
  \lambda_2 P\left((i_s;i_q),(n-i_s;j_q-1),R_2,2,2\right) +
  \lambda_{1,2} P\left((i_s;i_q),(n-i_s;j_q),R_2,1,2\right) + \\ +
  \mu_{1,2}  \delta\left(R_1(i_s+i_q; n-i_s+j_q)\right)
  P\left((i_s;i_q),(n-i_s;j_q),R_1(i_s+i_q; n-i_s+j_q),2,1\right) + \\ +
  i_s p_1\mu_2  P\left((i_s;i_q+1),(n-i_s;j_q),R_2,2,2\right) + \\ +
  (n-i_s) p_2 \mu_2  P\left((i_s;i_q),(n-i_s;j_q+1),R_2,2,2\right) +
  \\ +
  (i_s+1) p_2 \mu_2  P\left((i_s+1;i_q),(n-i_s-1;j_q+1),R_2,2,2\right)
  + \\ +
  (n-i_s+1) p_1\mu_2 P\left((i_s-1;i_q+1),(n-i_s+1;j_q),R_2,2,2\right), \\
  i_s=\overline{1,n-1},\quad  i_q+j_q\geqslant  1.
\label{eq:10.4}
\end{multline}
Here $p$ --- the probability that the first type application is taken
from the buffer, $p_2$ --- the probability that the second type
application is taken from the buffer.



%%%%%%%%%%%%%%%%%
\section{Conclusions}

The mathematical model of the system with the allocation of resources
to incoming applications and functioning in the random environment is
constructed. The system of equations for steady-state probability
distribution of the random process, which describes the functioning of
the system, is present.

The main task of future research is to present this system of
equations in a matrix form and try to apply the well known matrix
algorithms~\cite{Neuts4,Neuts5,MMAP-book1,MMAP-book2,MMAP-book4} in
order to obtain the steady-state probability distribution in the
analitycal form.

Also of interest are stationary distributions of applications of each
type, the average value of the system's system resources, the average
number of discarded (lossed) applications.


\begin{acknowledgments}
The publication has been prepared with the support of the
  ``RUDN University Program 5-100'' and funded by RFBR according to
  the research projects No.~18-07-00692 and No.~16-07-00766.
\end{acknowledgments}


\putbib[cite]


} % END \selectlanguage


%%% Local Variables:
%%% mode: latex
%%% coding: utf-8-unix
%%% End:

